% Usenix submission:
%   13-pages, excluding bibliography and well-marked appendices
%   (no more than 16 pp. long)


\documentclass[letterpaper,twocolumn,10pt]{article}
\usepackage{usenix}
\usepackage[T1]{fontenc}
\usepackage[%
    breaklinks=true,colorlinks=true,linkcolor=black,%
     citecolor=black,urlcolor=black,bookmarks=true,bookmarksopen=false,%
    pdfauthor={},
    pdftitle={DDosCoin}
    ,pdftex]{hyperref}

\usepackage{amsmath,amssymb,amsbsy}
\usepackage{wasysym}
\usepackage{longtable}
\usepackage{mathptmx}\renewcommand{\ttdefault}{cmtt}
\usepackage[margin=0pt]{caption}
\usepackage{graphicx}
\usepackage{tikz}
\usepackage{xcolor}
\usepackage{tablefootnote}
\usepackage{longtable}
\usepackage{color}
\usepackage{url}\urlstyle{rm}
\usepackage[margin=1in]{geometry}
\usepackage{watermark}
\usepackage{booktabs}
\usepackage{listings}
\usepackage{cite}
\usepackage{paralist}
\usepackage{balance}
\usepackage{tabularx}
\usepackage[protrusion=true,expansion=true,kerning]{microtype}
\usepackage{algorithm, algorithmic}
\usepackage{multirow}
\usepackage{placeins}
\usepackage{pgfplots}
%\usepackage{wrapfig}
\usepackage{colortbl}
\usepackage{bigstrut}
\usepackage{subcaption}
\newcommand{\specialcell}[2][c]{%
  \begin{tabular}[#1]{@{}c@{}}#2\end{tabular}}
\usepackage{float}

\newcommand{\ind}{\hspace*{1em}}

%\hyphenpenalty=62

\marginparwidth 36pt
\newcommand{\mn}[1]{\marginpar{\scriptsize \raggedright #1}}

% Paragraph and subpar
\renewcommand{\paragraph}[1]{\medskip\noindent\textbf{#1}\quad}
\newcommand{\subpar}[1]{\medskip\noindent\textsl{#1}\enspace}

% Fix ugly USENIX subsection headings (AH 3/08)
\makeatletter
\renewcommand{\section}{\@startsection {section}{1}{\z@}%
                                   {-3.5ex plus-1ex minus -.2ex}%
                                   {2.3ex plus.2ex}%
                                   {\normalfont\large\bfseries}}
\renewcommand{\subsection}{\@startsection{subsection}{2}{\z@}%
                                     {-2.5ex plus-.7ex minus -.2ex}%
                                     {1.5ex plus .2ex}%
                                     {\normalfont\fontsize{11}{12.5}\bfseries}}
\makeatother

\clubpenalty10000

% Stop URLs from hyphenating after  "http:" (AH 12/08)
\def\UrlBreaks{\do-\do\.\do\@\do\\\do\!\do\_\do\|\do\;\do\>\do\]%
 \do\)\do\,\do\?\do\'\do+\do\=\do\#}
\def\UrlBigBreaks{\do\:\do\/}%

% TODO, TK, etc. (AH 4/12)
\usepackage{xspace}
\newcommand{\todo}[1]{{\color{red}{\textbf{\em [TODO: #1]}}}\xspace}
\newcommand{\TODO}[1]{\todo{#1}}
\newcommand{\tk}{{\color{red}{\bf TK}}\xspace}
\newcommand{\TK}{\tk}
\newcommand{\comment}[1]{\relax} % comment out text
\newcommand{\xcite}[1]{\relax} % comment out citation

    %\thiswatermark{\parbox{\textwidth}{\vskip30pt\centering
    %\vspace*{-20pt}%
    % This paper appeared in \emph{Proceedings of the 9th {\small USENIX} Workshop on Offensive Technologies (WOOT)}, August~2015.\\
    % Source code and an online demonstration are available at \href{https://keysforge.com/}{{\bf https://keysforge.com/}}.\\
    %\vskip6pt
    %\rule[\baselineskip]{\textwidth}{0.5pt}
    %}}


% Poor UTF support in LaTeX, stolen from http://tex.stackexchange.com/questions/279100/typeset-the-shrug-%C2%AF-%E3%83%84-%C2%AF-emoji
\newcommand{\shrug}[1][]{%
\begin{tikzpicture}[baseline,x=0.8\ht\strutbox,y=0.8\ht\strutbox,line width=0.125ex,#1]
\def\arm{(-2.5,0.95) to (-2,0.95) (-1.9,1) to (-1.5,0) (-1.35,0) to (-0.8,0)};
\draw \arm;
\draw[xscale=-1] \arm;
\def\headpart{(0.6,0) arc[start angle=-40, end angle=40,x radius=0.6,y radius=0.8]};
\draw \headpart;
\draw[xscale=-1] \headpart;
\def\eye{(-0.075,0.15) .. controls (0.02,0) .. (0.075,-0.15)};
\draw[shift={(-0.3,0.8)}] \eye;
\draw[shift={(0,0.85)}] \eye;
% draw mouth
\draw (-0.1,0.2) to [out=15,in=-100] (0.4,0.95); 
\end{tikzpicture}}


\pagestyle{plain}
\thispagestyle{empty}
\begin{document}

\title{DDoSCoin: Cryptocurrency with a Malicious Proof-of-Work}

\author{\rm{Eric Wustrow}\\
University of Colorado, Boulder\\
{\small ewust@colorado.edu}
}

\maketitle




\begin{abstract}

Since its creation in 2009, Bitcoin has used a hash-based proof-of-work to
generate new blocks, and create a single public ledger of transactions. The
hash-based computational puzzle employed by Bitcoin is instrumental to its
security, preventing sybill attacks and making double-spending attacks more
difficult. However, there have been concerns over the efficiency of this
proof-of-work puzzle, and alternative ``useful'' proofs have been proposed.

In this paper, we present DDoSCoin, which is a cryptocurrency with a
\emph{mallicious} proof-of-work. DDoSCoin allows miners to prove that they have
contributed to a distributed denial of service attack against a specific target.
This proof involves making a large number of TLS connections to a target server,
and using infrequent cryptographic responses as a proof.



\end{abstract}




\section{Introduction}

Cryptocurrencies rely on proofs-of-work (PoW) that
require miners to spend a large amount of effort to solve a specific puzzle,
such that once solved, allowing others to inexpensively verify the solution is a
valid one. In Bitcoin, the proof-of-work is a computational puzzle based on the
SHA256 hash function, and miners are tasked with finding a partial preimage. The
only known way to do this is to iterate over a large number of inputs to the
hash function, and check if the output hash is less than a specified target. In
simplified terms, to find an input that hashes to N bits of 0s, the miner must
perform on average $2^{N}$ hashes. However, once found, other miners and Bitcoin
nodes can verify the puzzle solution with
a single hash~\footnote{Bitcoin uses double-SHA256 in its design, but this
detail is not important for the purposes of this paper.}.


Although the proof-of-work used in Bitcoin gives it resistance to sybil attacks,
the amount of computational effort carried out collectively by its miners does
not contribute to any problems besides securing the currency from attack.
Indeed, this computational effort is substantial: As of 2016, miners
collectively perform roughly $10^{18}$ (or about $2^{60}$) hashes every
second~\cite{blockchain}.

Previously, researchers have proposed alternative cryptocurrencies (altcoins)
that aim to have more beneficial proofs-of-work~\cite{primecoin,permacoin} that
provide utility beyond securing the underlying currency. In this paper however,
we investigate going in the opposite direction, and propose an altcoin that has
a \emph{malicious} proof-of-work that is externally detrimental.

In particular, we propose a proof-of-work that allows miners to prove they have
participated in a distributed denial of service (DDoS) attack against a
particular target. Miners are incentivized to send and receive large amounts of network
traffic at the target in order to produce a valid proof-of-work. As in other
cryptocurrencies, these proofs can be inexpensively verified by others, and the
original miner can collect a reward. This reward can be sold for other
currencies (including Bitcoin or even traditional currencies), allowing botnet
owners and other attacks to directly collect revenue for their assistance
in a decentralized DDoS attack.


The malicious proof-of-work operates by having miners create a large number of
TLS connections to a target webserver, and using the server's signed responses
as a proof of connection. In modern versions of TLS, the server signs a
client-provided parameter of the connection, along with server-provided values
used in the key exchange of the connection. This allows the client to prove to
others that it has communicated with the server. In addition, the signed value
returned by the server is not predictable to the client, and is randomly
distributed. Thus, clients can use a similar trick as in Bitcoin, and only
report connections that match some rare threshold, such as the signature from
the server starts with $N$ bits of 0s. On average, it will take clients $2^{N}$
connections to produce such a proof.

Although the malicious proof-of-work only works against websites that support
TLS 1.2, as of April 2016, over 51\% of the Alexa top million websites support
this version of TLS~\cite{censys}. Furthermore, we expect this number to
increase as TLS support becomes more widespread~\cite{letsencrypt}.

\emph{Contributions}
\begin{itemize}
\item Propose a novel conceptual cryptocurrency, DDoSCoin, whose proof-of-work
incentives miners to participate in a DDoS attack, and prove they have done so.
\item Implement our proof-of-work function and evaluate its performance.
\item Discuss and estimate the impact such a cryptocurrency would have if
deployed, and outline defenses.
\end{itemize}

Section blah does blah, and Section blat describes blat.





\section{Related Work}

Ethereum
Permacoin
Primecoin
Other interesting proofs-of-work?

Botnet-as-a-service

\section{Background}

\section{DDoSCoin Design}

\section{Discussion}

-Proof of (burnt) bandwidth
-Defenses
-Target server's participation


\section{Conclusion}



{\footnotesize\balance
\bibliographystyle{plain}
\bibliography{paper}}

\end{document}
